\begin{longtable}{|p{.3\linewidth}|p{0.6\linewidth}|}
Testfall & Zu testende Komponte / Funktion \\\hline

\textit{/T50/ Achterbahn visualisieren} & \textit{Achterbahn} \\\hline

\textit{/T100/ Konstruktion "offnen/schlie"sen} & \textit{RollercoasterFrame} \\\hline

\textit{/T200/ Starten/Stoppen der Simulation} & \textit{RollercoasterFrame} \\\hline

\textit{/T300/ Pausieren der Simulation} &  \textit{RollercoasterFrame}\\\hline

\textit{/T400/ Beenden des Programms} & \textit{RollercoasterFrame}\\\hline

\textit{/T510/ Einstellung des Graphen anpassen} &  \textit{RollercoasterFrame}\\\hline

\textit{ /T530/ Grafikeinstellungen "andern} &  \textit{RollercoasterFrame}\\\hline

\textit{/T531/ Neuanordnung Interface} &  \textit{RollercoasterFrame} \\\hline

\textit{/T532/ HUD ein/abschalten} & \textit{HUD}\\\hline

\textit{/T533o/ Kameraperspektive "andern} &  \textit{Getestet wird die komplette Klasse CameraControl}\\\hline

\textit{/T1000/ Berechnung der Bezierkurve}  &  \textit{BezierCurve}\\\hline

\textit{ /T 1001/ L"ange der Bezierkurve} &  \textit{getLength()}\\\hline

\textit{ /T 1002/ Punkt auf der Bezierkurve} &  \textit{getPoint(position)}\\\hline

\textit{ /T 1003/ Interpolation} &  \textit{cubicInterpolation(p0, p1, p2, p3, s)} \\\hline

\textit{ /T 1004/ Interpolation zwischen Vektoren} &  \textit{getInterpolation(p0, p1, p2, p3, s)}\\\hline

\textit{ /T 1005/ Interpolation zwischen Skalaren} &  \textit{getInterpolation(p0, p1, p2, p3, s)}\\\hline

\textit{ /T 1006/ Berechnung der Ableitung} &  \textit{getDerivative(p0, p1, p2, p3, s}\\\hline

\textit{ /T 1007/ Berechnung der 2. Ableitung} &  \textit{getSecondDerivative(p0, p1, p2, p3, s}\\\hline

\textit{ /T 1008/ Berechnung der Punktsequenz} &  \textit{getPointSequence(maxDistance, maxAngle)}\\\hline

\textit{/T1100/ LU-Zerlegung}  &  \textit{LUDecomposition}\\\hline

\textit{/T1200/ Vektorrechnung} &  \textit{Vector3d}\\\hline

\textit{/T 1201/ Skalare Multiplikation} & \textit{mult(scalar)}\\\hline

\textit{ /T 1202/ Skalare Division} &  \textit{div(scalar)}\\\hline

\textit{ /T 1203/ Vektorl"ange} &  \textit{length()}\\\hline

\textit{ /T 1204/ L"angenquadrat} &  \textit{lengthSquared()}\\\hline

\textit{ /T 1205/ Vektoraddition} &  \textit{add(other)}\\\hline

\textit{ /T 1206/ Vektorsubtraktion} &  \textit{add(other)}\\\hline

\textit{ /T 1207/ Inneres Produkt} &  \textit{dot(other)}\\\hline

\textit{ /T 1208/ Kreuzprodukt} &  \textit{cross(other)}\\\hline

\textit{ /T 1209/ Normierung} & \textit{normalize()}\\\hline

\textit{ /T 1210/ Flie"skommakonvertierung} &  \textit{toF()}\\\hline

\textit{ /T 1211/ Winkel zwischen Vektoren}  &  \textit{cos(x,y)}\\\hline

\textit{ /T 1212/ Gleichheit von Vektoren} &  \textit{equals(other), hashCode()}\\\hline

 \textit{/T1300/ RungeKutta} &  \textit{RungeKutta}\\\hline

\textit{/T1400/ Berechnung der Bahnkurve} &  \textit{RollercoasterTrajectory}\\\hline

\textit{ /T 1401/ Bestimmung des Bahnkurvenpunktes} &  \textit{getState()}\\\hline

\textit{ /T 1402/ Berechnung der Zeitprogession} & \textit{computeTimeStep(deltaTime), getState()}\\\hline

\textit{ /T 1403/ Berechnung der Ableitungen} &  \textit{getDerivatives(t,x)}\\\hline



\end{longtable}