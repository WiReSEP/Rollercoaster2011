% Kapitel 3 mit den entsprechenden Unterkapiteln
% Die Unterkapitel können auch in separaten Dateien stehen,
% die dann mit dem \include-Befehl eingebunden werden.
%----------------------------------------------------------------------------

\chapter{Testdurchführung}

In diesem Abschnitt werden die einzelnen Testfälle beschrieben und deren
Durchführungen (=Testläufe) protokolliert.\\
Ein Testfall ist eine Kombination von Eingabedaten, Bedingungen und erwarteten
Ausgaben, die einen bestimmten Zweck erfüllen. Man prüft z.B., ob Vorgaben in
einem Spezifikations-dokument eingehalten werden oder ob der Programmablauf
tatsächlich dem erwarteten Pfad entspricht.\\
Dieses Kapitel enthält drei Arten von Tabellen:\\
1.  Die Übersichtstabelle zeigt an, welche Testfälle es gibt und welcher
Testfall welche Objekte, Methoden oder Anforderungen testet. So hat man den
Überblick, Verfolg-barkeit zwischen der Testdokumentation und anderen
Dokumenten, und man kann sehen, ob die Testfälle vollständig sind.\\
2.  Der Testfall beschreibt jeden einzelnen Testfall im Detail.\\
3.  Der Testlauf beschreibt eine Durchführung eines Testfalls. Derselbe
Testfall kann mit verschiedenen Eingabedaten oder auch mit verschiedenen
Softwareversionen mehrmals durchgeführt werden.\\

\section{Übersichtstabelle}
  In der folgenden Tabelle sind entweder für alle Testfälle die zu testende
  Komponente oder die zu testende Funktion angegeben werden. Oder beides. Die
  Bezeichnungen der Komponenten müssen konsistent sein mit denen in Fein- und
  Grobkonzept, um die Verfolgbarkeit zum Konzept sicherzustellen. Die IDs und
  Bezeichnungen der Funktionen müssen denen im Pflichtenheft entsprechen, um
  die Verfolgbarkeit zu den Anforderungen sicherzustellen. \\
\begin{tabular}{|c|c|c|}
\hline
\textbf{Testfall ID und Bezeichnung} &  \textbf {Zu testende Komponente} &
\textbf {Zu testende Funktion}\\
\hline
z.B. /T100/ Lager anlegen &  z.B. Lagerdatenbank  & z.B. /F100/ Lager anlegen \\
\hline
&&
\end{tabular}

Im Folgenden sind so viele Unterkapitel einzufügen, wie es Testfälle gibt.\\

\section{Testfall -- ID und Bezeichnung}
Jeder Testfall erhält eine eindeutige Identifikation mit Kurzbezeichnung.\\
Beispiel: /T100/ Lager anlegen\\
Die folgende Tabelle beschreibt den Testfall. \\
\begin{longtable}{|p{7cm}|p{10cm}|}
\hline
\textbf{Testfall -- ID und Bezeichnung} &  \textit{Beispiel:
                                                        /T100/ Lager anlegen} \\
\hline
\textbf{Zu testende Objekte und Methoden} &  \textit{Hier sind alle Testobjekte
und Methoden zu beschreiben, die von diesem Testfall ausgeführt werden.
Testobjekte können dabei z.B. auch Komponenten oder einzelne Webseiten sein.}
\\
\hline
\textbf{Kriterien für erfolgreiche bzw. fehlgeschlagene Testfälle} &
\textit{Es sind die Kriterien anzugeben, mit denen man feststellt, dass der
Testfall erfolgreich bzw. fehlgeschlagen ist. } \\
\hline
\textbf{Einzelschritte} &  \textit{Es ist zu beschreiben, was zu tun ist, um
einen Testlauf vorzubereiten und ihn zu starten.
Ggf. sind erforderliche Schritte während seiner Ausführung anzugeben (z.B.
Benutzerinteraktion über ein User-Interface). Ferner ist zu beschreiben, was zu
tun ist, um den Testlauf ordnungsgemäß oder im Falle unvorhergesehener
Ereignisse anzuhalten (falls er nicht von selbst terminiert).
Ggf. sind Aufräumarbeiten zu beschreiben, um nach den Tests den ursprünglichen
Zustand wiederherzustellen (falls der Testlauf nicht seiteneffektfrei ist)
} \\
\hline
\textbf{Beobachtungen / Log} &  \textit{Es sind alle speziellen Methoden oder
Formate zu beschreiben, mit denen die Ergebnisse der Testläufe, die
Zwischenfälle und sonstige wichtige Ereignisse aufgenommen werden sollen.
Beispiel: Logdatei eines Servers, Messung der Antwortzeit eines Remote
Terminals mittels Netzwerk Simulator, \ldots} \\
\hline
\textbf{Besonderheiten } &  \textit{optional; auszufüllen, falls es
Besonderheiten in diesem Testfall gibt.
Testfallspezifische Besonderheiten, z.B. Ausführungsvorschriften oder
Abweichungen von der Testumgebung (siehe 2.5)  werden hier aufgelistet.} \\
\hline
\textbf{Abhängigkeiten} &  \textit{optional; auszufüllen, falls es
Abhängigkeiten in diesem Testfall gibt
Ist dieser Testfall von der Ausführung anderer Testfälle abhängig, so werden
diese Testfälle hier aufgelistet und kurz beschrieben, worin die Abhängigkeit
besteht.} \\
\hline

 \end{longtable}

Die folgenden Tabellen beschreiben, wie der Testfall ausgeführt wurde und
welches Ergebnis er geliefert hat. Da es bei Korrektur von Softwarefehlern oder
anderen Gegebenheiten notwendig ist, einen Test mehrfach durchzuführen
(Testläufe), ist jede Testdurchführung zu dokumentieren. Daher ist diese
Tabelle für \textbf{jeden Testlauf }  zu erstellen und \textbf{ fortlaufend zu
nummerieren}. \\


\begin{longtable}{|p{7cm}|p{10cm}|}
\hline
\textbf{Testfall -- ID und Bezeichnung} & \textit{Beispiel: /T100/ Lager
anlegen} \\
\hline
\textbf{Testlauf Nr.} & \textit{Beispiel: 1} \\
\hline
\textbf{Eingaben} & \textit{Es sind alle Eingabedaten bzw. andere Aktionen
aufzufüh-ren, die für die Ausführung des Testfalls notwendig sind.
Diese können sowohl als Wert angegeben werden (ggf. mit Toleranzen) als auch
als Name, falls es sich um konstante Tabellen oder um Dateien handelt. Außerdem
sind alle betroffenen Datenbanken, Dateien, Terminal Meldungen, etc. anzugeben.
Hinweis: Es sind nicht noch mal die Einzelschritte aus 3.1.3 zu wiederholen.
Während jene allgemeiner sind (z.B. "`Ein-loggen über das Login-Formular"')
sind hier die konkreten eingegebenen Testdaten aufzuführen (z.B. "`Loginname:
test; Passwort: xxxtest"'`). } \\
\hline
\textbf{Soll - Reaktion} & \textit{Hier ist anzugeben, welches Ergebnis bzw.
Ausgabe der Test haben soll.
Hinweis: Es sind nicht noch mal die Erfolgskriterien aus 3.1.2 zu wiederholen.
Während jene allgemeiner sind (z.B. "`Testnachricht wird über Netzwerkkanal
empfangen"') sind hier die konkreten erhaltenen Testdaten aufzuführen (z.B.
Konsole zeigt Meldung: "`Testnachricht 123 erhalten"').
} \\
\hline
\textbf{Ist -- Reaktion} & \textit{Hier ist anzugeben, welches Ergebnisdaten
bzw. Ausgaben dieser Testlauf geliefert hat.} \\
\hline
\textbf{Ergebnis} & \textit{Für jeden Testlauf ist zu vermerken, ob der Test
erfolgreich durchgeführt werden konnte oder nicht. Einen missglückten Test
bitte begründen, sofern der Grund des Fehlschlags bekannt oder offensichtlich
ist.} \\
\hline
\textbf{Unvorhergesehene Ereignisse während des Test-laufs } &
\textit{optional; nur anzugeben, falls es unvorhergesehene Ereig-nisse gab} \\
\hline
\textbf{Nacharbeiten } & \textit{Ist ein Testlauf nicht erfolgreich
durchgeführt worden, so werden hier die erforderlichen Nacharbeiten aufgeführt
(z.B. Bugfixes).} \\
\hline
 \end{longtable}




% %%%%%%%%%%%%%%%%%%%%%%%%%%

\begin{longtable}{|p{7cm}|p{10cm}|}
\hline
\textbf{Testfall -- ID und Bezeichnung} &  \textit{Beispiel:
                                                        /TXYZ/ Achterbahn visualisieren} \\
\hline
\textbf{Zu testende Objekte und Methoden} &  \textit{Achterbahn}
\\
\hline
\textbf{Kriterien für erfolgreiche bzw. fehlgeschlagene Testfälle} &
\textit{In 5 Schritten wird eine Achterbahn gezeigt, die immer mehr Details zur Verfügung stellt. Dabei treten keine Ladefehler auf. Die am Schluss dargestellte Achterbahn entspricht den Spezifikiationen. } \\
\hline
\textbf{Einzelschritte} &  \textit{ Die Klasse AchterbahnTest muss mit den Bibliotheken der jMonkeyEngine kompiliert und dann ausgeführt werden. Durch Drücken der Leertaste werden die 5 Fälle durchgeschaltet bis das 
Programm schließlich terminiert. Im Fehlerfall ist sicherzustellen, dass die verwendeten Bibliotheken zur 3D-Beschleunigung freigegeben wurden.
} \\
\hline
\textbf{Beobachtungen / Log} &  \textit{Das Programm öffnet ein Fenster indem eine 3D-Anzeige stattfindet. In dem Fenster kann mit der Maus und den Tasten W,A,S,D navigiert werden um die Geometry näher zu betrachten \ldots} \\
\hline
\textbf{Besonderheiten } &  \textit{Da es sich um eine visuelle Komponente handelt, ist ein automatisierter Test mit jUnit nicht möglich. Daher wird die Testlösung über eine kleine Application verwendet.} \\
\hline

 \end{longtable}




%%%%% Durchläufe  %%%%%%%%%%%%%%%%%


\begin{longtable}{|p{7cm}|p{10cm}|}
\hline
\textbf{Testfall -- ID und Bezeichnung} & \textit{ /TXYZ/ Achterbahn visualisieren} \\
\hline
\textbf{Testlauf Nr.} & \textit{1} \\
\hline
\textbf{Eingaben} & \textit{} \\
\hline
\textbf{Soll - Reaktion} & \textit{Nach dem Druck auf die Leertaste wird eine Geometry angezeigt in der vier Eckpunkte eines Quadrats mit zylindrischen Zwischenstücken verbunden sind. Die Ecken sind abgeschrägt. Einfache Rechtecke liegen in den Eckpunkten.
Nach Druck auf die Leertaste verändert sich die Art der Verbindung zu einem angegebenen Querschnitt (Standartachterbahnprofil skaliert). Ein weiterer Druck auf die Leertaste ersetzt die Rechtecke in den Eckpunkten durch 
das Standartmodell für Joints. Ein weiterer Druck auf die Leertaste ändert die Art der Oberflächenfärbung gemäß der Materialeigenschaften aus der Modelldatei. Ein weiterer Druck auf die Leertaste lädt die Beispielachterbahn Colossos und stellt diese dar.
} \\
\hline
\textbf{Ist -- Reaktion} & \textit{Die erwarteten 3D-Objekte werden angezeigt, jedoch sieht man die Innenseite der zylindrischen Verbindungen. Die Boxen bzw Joints werden nicht korrekt senkrecht zur Bahn eingefügt.} \\
\hline
\textbf{Ergebnis} & \textit{Die Teilfunktionalität des Extrudierens kann als funktionierend betrachtete werden. Der Test ist missglückt, da es offensichtlich zu Fehlern bei der Berechnung beim Clipping kommt. 
Für die korrekte Rotation der Joints muss zwischen Rechts- und Linkssystem konvertiert werden.} \\
\hline
\textbf{Nacharbeiten } & \textit{Es müssen Abhängig von der Richtung und räumlichen Orientierung die Reihenfolge der Dreieckseckpunkte angepasst werden um dem OpenGl-Standart der CCW-Dreiecksübergabe zu genügen.
Für die Konvertierung muss die RollAchse invertiert werden.} \\
\hline
 \end{longtable}


\begin{longtable}{|p{7cm}|p{10cm}|}
\hline
\textbf{Testfall -- ID und Bezeichnung} & \textit{ /TXYZ/ Achterbahn visualisieren} \\
\hline
\textbf{Testlauf Nr.} & \textit{2} \\
\hline
\textbf{Eingaben} & \textit{} \\
\hline
\textbf{Soll - Reaktion} & \textit{Nach dem Druck auf die Leertaste wird eine Geometry angezeigt in der vier Eckpunkte eines Quadrats mit zylindrischen Zwischenstücken verbunden sind. Die Ecken sind abgeschrägt. Einfache Rechtecke liegen in den Eckpunkten.
Nach Druck auf die Leertaste verändert sich die Art der Verbindung zu einem angegebenen Querschnitt (Standartachterbahnprofil skaliert). Ein weiterer Druck auf die Leertaste ersetzt die Rechtecke in den Eckpunkten durch 
das Standartmodell für Joints. Ein weiterer Druck auf die Leertaste ändert die Art der Oberflächenfärbung gemäß der Materialeigenschaften aus der Modelldatei. Ein weiterer Druck auf die Leertaste lädt die Beispielachterbahn Colossos und stellt diese dar.
} \\
\hline
\textbf{Ist -- Reaktion} & \textit{Bis zur Colossos-Bahn entspricht die Ist-Reaktion den Erwartungen. Bei der Colossos-Bahn bricht die Bahn mitten im Verlauf ab} \\
\hline
\textbf{Ergebnis} & \textit{Offensichtlich existiert eine Hardwaregrenze die es verbietet, mehr als $2^{16}$ Dreiecke in einem Framebufferobject zu hinterlegen. Der Fehler ist somit Hardwareseitig.} \\
\hline
\textbf{Nacharbeiten } & \textit{Anpassen der Grundlage für die dynamische Generierung, sodass weniger Dreiecke entstehen} \\
\hline
 \end{longtable}

\begin{longtable}{|p{7cm}|p{10cm}|}
\hline
\textbf{Testfall -- ID und Bezeichnung} & \textit{ /TXYZ/ Achterbahn visualisieren} \\
\hline
\textbf{Testlauf Nr.} & \textit{3} \\
\hline
\textbf{Eingaben} & \textit{} \\
\hline
\textbf{Soll - Reaktion} & \textit{Nach dem Druck auf die Leertaste wird eine Geometry angezeigt in der vier Eckpunkte eines Quadrats mit zylindrischen Zwischenstücken verbunden sind. Die Ecken sind abgeschrägt. Einfache Rechtecke liegen in den Eckpunkten.
Nach Druck auf die Leertaste verändert sich die Art der Verbindung zu einem angegebenen Querschnitt (Standartachterbahnprofil skaliert). Ein weiterer Druck auf die Leertaste ersetzt die Rechtecke in den Eckpunkten durch 
das Standartmodell für Joints. Ein weiterer Druck auf die Leertaste ändert die Art der Oberflächenfärbung gemäß der Materialeigenschaften aus der Modelldatei. Ein weiterer Druck auf die Leertaste lädt die Beispielachterbahn Colossos und stellt diese dar.
} \\
\hline
\textbf{Ist -- Reaktion} & \textit{Die Darstellung entspricht den Erwartungen} \\
\hline
\textbf{Ergebnis} & \textit{Test erfolgreich abgeschlossen.} \\
\hline
\textbf{Nacharbeiten } & \textit{} \\
\hline
 \end{longtable}

\begin{longtable}{|p{7cm}|p{10cm}|}
\hline
\textbf{Testfall -- ID und Bezeichnung} &  \textit{/T150/ Konstruktion öffnen/schließen } \\
\hline
\textbf{Zu testende Objekte und Methoden} &  \textit{RollercoasterFrame}
\\
\hline
\textbf{Kriterien für erfolgreiche bzw. fehlgeschlagene Testfälle} &
\textit{Es können verschiedene Konstruktionen aus dem gestellten Editor geöffnet werden. Dabei werden die Daten vom XML-Loader geladen, der Streckenverlauf von der Physik berechnet
 und von der 3D-Engine interpretiert. Die 3D-Darstellung der Bahn wird im Canvas Container initialisiert. Beim schliessen werden alle Daten verworfen und die 3D-Darstellung wird beendet.} \\
\hline
\textbf{Einzelschritte} &  \textit{Die Klasse RollercoasterFrame muss mit den restlichen Komponenten des Programms kompiliert und ausgeführt werden. Durch die Benutzung der MenuItmes
Konstruktion öffnen/schliessen wird ein FileChooser geöffnet, mit dem die zu öffnende Datei angegeben werden kann. Ist die Datei ausgewählt, wird die 3D-Anzeige initialisiert und man befindet
sich am Startpunkt der Bahn mit Sicht aus dem Wagen.} \\
\hline
\textbf{Beobachtungen / Log} &  \textit{Es wird die 3D-Anzeige im Canvas Feld initialisiert und der Wagen steht an seinem Startpunkt.} \\
\hline
\textbf{Abhängigkeiten} &  \textit{Das Öffnen der Konstruktion ist abhängig vom XML-Loader und der Verarbeitung der daraus resultierenden Daten in der Physik und der 3D-Engine.} \\
\hline
 \end{longtable}


\begin{longtable}{|p{7cm}|p{10cm}|}
\hline
\textbf{Testfall -- ID und Bezeichnung} & \textit{/T150/ Konstruktion öffnen/schließen} \\
\hline
\textbf{Testlauf Nr.} & \textit{1} \\
\hline
\textbf{Eingaben} & \textit{Benutzung der Konstruktion öffnen/schliessen Buttons in beliebiger Kombination.} \\
\hline
\textbf{Soll - Reaktion} & \textit{ Die 3D-Anzeige soll fehlerfrei initialisiert werden.} \\
\hline
\textbf{Ist -- Reaktion} & \textit{ Alle Aktionen wurden wie erwartet ausgeführt.} \\
\hline
\textbf{Ergebnis} & \textit{Der Test wurde bestanden.} \\
\hline
 \end{longtable}




\begin{longtable}{|p{7cm}|p{10cm}|}
\hline
\textbf{Testfall -- ID und Bezeichnung} &  \textit{/T200/ Starten/Stoppen der Simulation aus der GUI} \\
\hline
\textbf{Zu testende Objekte und Methoden} &  \textit{RollercoasterFrame}
\\
\hline
\textbf{Kriterien für erfolgreiche bzw. fehlgeschlagene Testfälle} &
\textit{Durch die Benutzung der Startbuttons, sowohl aus der GUI-Oberfläche sowie aus dem Menü, soll im Canvas-Container die 3D-Darstellung der Achterbahn vom Ausgangspunkt gestartet werden.
Desweiteren wird im Log eine Nachricht ausgegeben, dass die Simulation gestartet worden ist. Der Graph und die Minima/Maxima Tabelle werden mit Daten befüllt und stetig aktualisiert.
Der Stoppbutton beendet die 3D-Anzeige, die Befüllung des Graphens und der Minima/Maxima Tabelle mit Daten und gibt im Log die Nachricht aus, dass die Simulation beendet wurde. Die Daten im Graph
und der Tabelle werden weiter gehalten, bis ein neuer Startbefehl kommt.} \\
\hline
\textbf{Einzelschritte} &  \textit{Die Klasse RollercoasterFrame muss mit den restlichen Komponenten des Programms kompiliert und ausgeführt werden. Durch die Benutzung der Start- Stoppbuttons
kann nun die Simulation begonnen und beendet werden.} \\
\hline
\textbf{Beobachtungen / Log} &  \textit{Es wird die 3D-Anzeige im Canvas Feld gestartet, bzw. gestoppt. Der Graph und die Tabelle werden mit Daten befüllt, die dann dort ausgegeben werden.} \\
\hline
\textbf{Abhängigkeiten} &  \textit{Die Start und Stopfunktionen müssen auch funktionieren, wenn sich die Simulation im Pausemodus befindet.} \\
\hline
 \end{longtable}



\begin{longtable}{|p{7cm}|p{10cm}|}
\hline
\textbf{Testfall -- ID und Bezeichnung} & \textit{ /T200 / Starten/Stoppen der Simulation aus der GUI} \\
\hline
\textbf{Testlauf Nr.} & \textit{1} \\
\hline
\textbf{Eingaben} & \textit{Benutzung der Start- Stopbuttons in beliebiger Kombination.} \\
\hline
\textbf{Soll - Reaktion} & \textit{Die 3D-Anzeige wird im Canvas Feld fehlerfrei ausgeführt und gestoppt. Im Log werden die Nachrichten Simulation gestartet, bzw. Simulation gestoppt ausgegeben.
Der Graph gibt die Werte für Geschwindigkeit und Beschleunigung über die Zeit aus. In der Minima/Maxima Tabelle werden sowohl die aktuellen Werte für die Beschleunigung und Geschwindigkeit mit 
Zeit- und Winkelangabe zum aktuellen Berechnungszeitpunkt als Zahl ausgegeben, sowie den Minimal- und Maximalwert der auf dem gesamten Verlauf der Bahn erreicht worden ist.
} \\
\hline
\textbf{Ist -- Reaktion} & \textit{Die Anzeige startet korrekt, der Graph und die Tabelle werden mit den richtigen Daten befüllt. Nach mehrmaligem drücken auf den Startbutton stellt sich allerdings 
eine Erhöhung der Simulationsgeschwindigkeit ein. Desweiteren muss nach mehrmaligem drücken des Startbuttons, der Stopbutton genauso häufig gedrückt werden, bis die Anzeige tatsächlich beendet wird.
Das gleiche gilt in die andere Richtung. Wird zu Beginn der Stopbutton mehrmals gedrückt, muss der Startbutton ebenso häufig gedrückt werden, damit die Simualtion gestartet wird.} \\
\hline
\textbf{Ergebnis} & \textit{Der Test wurde nicht bestanden, da bei jedem Druck auf den Startbutton ein neuer Thread der 3D-Anzeige erzeugt wurde, die wiederum alle einzeln abgebrochen werden mussten.} \\
\hline
\textbf{Nacharbeiten } & \textit{Zu den Aktionen der Buttons wurde eine Abfrage hinzugefügt, ob schon ein Thread aktiv ist, oder nicht.} \\
\hline
 \end{longtable}



\begin{longtable}{|p{7cm}|p{10cm}|}
\hline
\textbf{Testfall -- ID und Bezeichnung} & \textit{ /T200 / Starten/Stoppen der Simulation aus der GUI} \\
\hline
\textbf{Testlauf Nr.} & \textit{2} \\
\hline
\textbf{Eingaben} & \textit{Benutzung der Start- Stopbuttons in beliebiger Kombination.} \\
\hline
\textbf{Soll - Reaktion} & \textit{Die 3D-Anzeige wird im Canvas Feld fehlerfrei ausgeführt und gestoppt. Im Log werden die Nachrichten Simulation gestartet, bzw. Simulation gestoppt ausgegeben.
Der Graph gibt die Werte für Geschwindigkeit und Beschleunigung über die Zeit aus. In der Minima/Maxima Tabelle werden sowohl die aktuellen Werte für die Beschleunigung und Geschwindigkeit mit 
Zeit- und Winkelangabe zum aktuellen Berechnungszeitpunkt als Zahl ausgegeben, sowie den Minimal- und Maximalwert der auf dem gesamten Verlauf der Bahn erreicht worden ist.
} \\
\hline
\textbf{Ist -- Reaktion} & \textit{Die Anzeige startet korrekt, der Graph und die Tabellen werden mit den richtigen Daten befüllt und auch die Reaktionen der Buttons sind nun so, wie man es zu
erwarten hat.} \\
\hline
\textbf{Ergebnis} & \textit{Der Test wurde bestanden.} \\
\hline
 \end{longtable}