% Kapitel 2 mit den entsprechenden Unterkapiteln
% Die Unterkapitel können auch in separaten Dateien stehen,
% die dann mit dem \include-Befehl eingebunden werden.
%----------------------------------------------------------------------------

\chapter{Testplan}


\section{Zu testende Komponenten}
Der Achterbahnsimulator ist eine Einzelplatzanwendung. Das fertige Produkt wird als monolitisches System abgenommen und ist deswegen im strengen Sinne keine verteile Anwendung. Da jedoch zu den Anforderungen an den Simulator das Einlesen der Achterbahndaten aus dem Editor zählt, muss dieses Zusammenspiel geprüft werden.

Der Achterbahnsimulator setzt sich gemäß Entwurf aus verschiedenen Komponenten zusammen, deren Aufteilung bei der Implementierung zur Beibehaltung der Konhärenz inenrhalb der Komponenten und Vermeidung von zwischen den Komponenten beizubehalten war.
Entsprechend können diese Komponenten getrennt geprüft werden. 

Da die Software eine 3D-Anzeige beeinhaltet ist es vor dem Test sicherzustellen, dass die Grafikkarte entsprechend inklusive 3D-Beschleunigung eingerichtet ist.
Die Software kann online (im Repository) oder lokal auf Festplatte oder CD vorliegen. In jedem Fall ist dafür Sorge zu tragen, dass die Software die Möglichkeit hat in dem lokalen Pfad Dateien anzulegen.

\section{Zu testende Funktionen}
Im folgenden findet sich eine Liste über die Funktionen des Programms die zu testen sind. 

/F100/ : Spezifikation einlesen\\
\\
/F200/ : Starten/Stoppen der Simulation\\
\\
/F300/ : Pausieren der Simulation\\
\\
/F500/ : Einstellungen ändern\\
\\
/F520/ : Simulationsparameter ändern\\
\\
/F530/ : Grafische Einstellungen ändern\\
\\
/F531/ : Neuanordnung (Interface)\\
\\
/F532/ : Ein-/Ausblenden von Beschleunigungsdaten\\
\\
/F533o/ : Kameraperspektive ändern\\
\\
/F1000/ : Warnung vor zu hoher Beschleunigung\\
\\
/F1100/ : Erkennung von Veränderungen an der Ursprungsdatei



\section{Nicht zu testende Funktionen}


\section{Vorgehen}

Während sich für den mathematisch-physikalischen Kern der Anwendung eine Prüfung mit automatisierten Tests umsetzen lässt, müssen diese für die visuellen Komponenten der GUI und 3D-Grafik weitestgehend händische vorgenommen werden.

Für die Durchführung der JUnit-Testfälle wurde ein \textbf{ant}-Task in der \textbf{build.xml} angelegt, dessen Prüfprotokolle als XML-Dateien generiert und nachbearbeitet werden können.

Für die Prüfung des 3D-Grafikmoduls wird eine reduzierte Variante des Simulators kompiliert und ausgeführt. Auch dafür steht ein passender \textbf{ant}-Task zur Verfügung.


\section{Testumgebung}


