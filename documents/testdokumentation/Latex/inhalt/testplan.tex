% Kapitel 2 mit den entsprechenden Unterkapiteln
% Die Unterkapitel können auch in separaten Dateien stehen,
% die dann mit dem \include-Befehl eingebunden werden.
%----------------------------------------------------------------------------

\chapter{Testplan}
Der Testplan ist das zentrale Dokument der Qualitätssicherung und wird daher
frühzeitig erstellt. Hier wird Umfang und Vorgehensweise der Qualitätssicherung
beschrieben. Außer-dem werden Testgegenstände und deren zu testenden
Eigenschaften bzw. Funktionen iden-tifiziert. Ferner werden die
durchzuführenden Maßnahmen und die dafür verantwortlichen Personen definiert.
Falls erforderlich sollte hier auch auf allgemeine Risiken eingegangen werden.

\section{Zu testende Komponenten}
Hier sind sämtliche zu testenden Objekte einschließlich der Versionsnummer bzw.
Versions-nummer aufzuführen. Ebenso ist anzugeben, auf welchem Medium die
Software vorliegt, ob dies einen Einfluss auf Hardwareanforderungen hat und ob
die Software vor Testbeginn in irgendeiner Weise transformiert werden muss
(z.B. ob sie von Tape auf Diskette kopiert wer-den muss). Außerdem wird auf zum
Objekt gehörende Dokumentation der Komponente (Lasten-, Pflichtenheft, Grob-
bzw. Feinentwurf) referenziert.

\section{Zu testende Funktionen}
Dieser Punkt beinhaltet alle Eigenschaften bzw. Funktionen und deren
Kombinationen, die zu testen sind.

Sämtliche Funktionalitäten, die getestet werden sollen, werden hier aufgeführt.
Dabei sind auf die vorangegangenen Dokumentationen zu referenzieren
(Pflichtenheft, Grob- und Fein-entwurf) und die dortigen Funktions-IDs zu
verwenden!\\ Beispiel: /F100/ : Benutzer Login


\section{Nicht zu testende Funktionen}
(optional; auszufüllen, falls es Funktionen gibt, die nicht getestet werden
sollen)\\

Hier werden alle Eigenschaften bzw. Funktionen und Funktionskombinationen
aufgelistet, die nicht getestet werden.
\textbf{ Es sollte begründet werden, warum diese nicht getestet werden.} Es
versteht sich von selber, dass alle Muss-Funktionalitäten des Pflichtenheftes
(Abschnitt 1.1) getestet werden müssen.

\section{Vorgehen}
Die allgemeinen Vorgehensweisen für die einzelnen zu testenden Funktionen und
Funkti-onskombinationen werden hier beschrieben. Die Beschreibung sollte
detailliert genug sein, um die Hauptaktivitäten und deren Zeitbedarf abschätzen
zu können.

Es ist zu beachten, dass für alle wichtigen Funktionalitäten das Verfahren
angegeben wird. Dies gewährleistet, dass diese Funktionalitäten adäquat
getestet werden.

Es ist zu dokumentieren, welche Aktivitäten, Techniken und Werkzeuge benötigt
werden, damit die Funktionalitäten getestet werden können.

Beispiel für Vorgehen (unvollständige Liste):\\
a) Komponenten- und Integrationstests\\
Klassen werden mit JUnit-Testfällen geprüft. Vor Beginn der Implementierung
werden bereits Blackbox-Testfälle erstellt, die dann begleitend zur
Implementierung genutzt werden ("`Test first"'). Nach Abschluss der
Implementierung einer Komponente wird diese dann durch Whitebox-Tests
geprüft.\\
Der Integrationstest der Klassen und Komponenten erfolgt nach dem
Bottom-Up-Prinzip. Anfangs muss die Integration der Datenbankanbindung und den
entsprechenden Data-Access-Objects (DAO) geprüft werden, da das Mapping der
Datenbank auf Objekte die unterste Schicht des Projektes bildet. Dieser
Testabschnitt wird durch die Schnittstellentests abgedeckt.
Die Komponenten werden damit unter Berücksichtigung ihrer Abhängigkeiten
konkret in folgender Reihenfolge integriert: \ldots\\
(Hier kommt das konkrete Vorgehen bei der Integration: Welche Klassen werden
zusammen getestet, welche kommen dann dazu etc. Das kann man z.B. auch schön in
Form eines Baumes aufzeigen.)
b) Funktionstests\\
Die Anwendungsfälle aus der Anforderungsspezifikation werden über das
Web-Interface geprüft. Mindestanforderung hierfür ist es, jeden Fall einmal auf
seine korrekte Funktionalität zu testen.\\
c) \ldots

\section{Testumgebung}
Die genutzte Testumgebung(en) bitte hier angeben und kurz beschreiben.\\
Beispiel: JUnit Testsuite, lokal installierter Web Application Server, \ldots

