% Kapitel 3 mit den entsprechenden Unterkapiteln
% Die Unterkapitel können auch in separaten Dateien stehen,
% die dann mit dem \include-Befehl eingebunden werden.
%------------------------------------------------------------------------------------
\chapter{Resultierende Softwarearchitektur}

Dieser Abschnitt hat die Aufgabe, einen Überblick über die zu entwickelnden
Komponenten und Subsysteme zu liefern.
\section{Komponentenspezifikation}

Aus der Analyse der Produktfunktionen ergeben sich sechs größere Teilbereiche,
die als Komponenten dieser Anwendung umgesetzt werden soll.

Im Kern der Anwendung steht die Simulator-Komponente, die für die Steuerung der
Achterbahnbewegung verantwortlich ist und zwischen der physikalischen Berechnung
und der grafischen Umsetzung koordiniert. Dieser Komponente wird von Außen die
Spezifikation einer Achterbahn übergeben, welche als Simulation umgesetzt werden
soll. Die Komponente hält über eine Endlosschleife (main loop) den Ablauf im Gang.

Die dreidimensionale Visualisierung der Achterbahn mit Gerüst und Umgebung wird
von der GraphicsEngine-Komponente übernommen. Sie kapselt die allgemeinen gehaltenen
Funktionen der eingesetzten Grafikbibliothek und stellt eine einfache Schnittstelle
für das Befüllen der virtuellen Welt mit Achterbahnkomponenten sowie zur Steuerung
der Kamerafahrten entlang der befahrenen Strecke zur Verfügung.

Die physikalische Berechnung der Achterbahnbewegung erfolgt in der PhysicsEngine-
Komponente. Hier wird die Bewegung eines Massenpunktes auf der vorgegebenen Raumkurve
nach den Gesetzen der klassischen Mechanik durch Lösen einer gewöhnlichen 
Differentialgleichung bestimmt. Die berechnete Bahnkurve wird dem Simulator zur
Umsetzung der Kamerafahrt bereitgestellt.

Um die über Stützstellen definierte Achterbahn in eine ununterbrochene und hinreichend
glatte Raumkurve umzurechnen, kommt die Komponente ``Mathematics'' zum Einsatz. Im
Wesentlichen besteht die Komponente aus den Routinen zur näherungsweisen Berechnung
von Beziérkurven.    

Für das Einlesen der Achterbahn-Spezifikationen aus den vom Editor bereitgestellten
XML-Dateien kommt die Komponente ``FileManagement'' zum Einsatz. Diese kapselt alle
am speziellen Serialisierung-Schema haftenden Eigenheiten ab und liefert einen
von den anderen Komponenten einfach auslesbaren Datensatz mit dem Stützstellen der
Bahn und weiteren Parametern.

Das GraphicalUserInterface übernimmt die Interaktion zwischen dem Benutzer und
der Simulation. Es übernimmt die Steuerung des Hauptmenüs, des Dateidialoges und
des Optionen-Dialogs. Die aus dem FileManagement ausgelesenen Achterbahn-Datensätze
werden an den Simulator zur Darstellung übergeben. Die Nutzeraktionen bezüglich des
Simulationsablaufes werden durchgereicht.

(Diagramm einfügen)

\section{Schnittstellenspezifikation}

Im Folgenden werden die einzelnen Schnittstellen der Komponenten aus der
Komponentenspezifikation näher erläutert, d. h. die von Ihnen zur Verfügung
gestellten Operationen werden dokumentiert. Die Tabelle ist dabei um so viele
Zeilen zu erweitern, wie es Schnittstellen im Komponentendiagramm gibt. In der
innen liegenden Aufteilung ist für jede Operation einer Schnittstelle eine
Zeile einzufügen.  Reine Set- und Get-Aufrufe brauchen nicht aufgeführt zu
werden (sollten auch möglichst nicht komponentenübergreifend auftauchen).

\begin{tabular}[ht]{|l|p{0.35\linewidth}|p{0.35\linewidth}|}
 \hline
 Schnittstelle & \multicolumn{2}{|c|}{Aufgabenbeschreibung}\\
 \hline\hline
    <Schnittstellen – ID>: & \multicolumn{2}{|c|}{Überschrift über zwei Spalten}\\
 \hline
 & Name der Funktion1 & Beschreibung1\\ 
 & Name der Funktion2 & Beschreibung2\\ 
 & Name der Funktion3 & Beschreibung3\\ 
 & Name der Funktion4 & Beschreibung4\\ 
\hline
    <Schnittstellen – ID>: & \multicolumn{2}{|c|}{Überschrift mal wieder}\\
 \hline
 & Name der Funktion & Beschreibung\\ 
 & Name der Funktion & Beschreibung\\ 
 & Name der Funktion & Beschreibung\\ 
 & Name der Funktion & Beschreibung\\ 
 \hline
   \end{tabular}





\section{Protokolle für die Benutzung der Komponenten}

In diesem Abschnitt wird mit Hilfe von Protokoll-Statecharts die korrekte
Verwendung der zu entwickelnden Komponenten dokumentiert. Dies ist insbesondere
für diejenigen Komponenten notwendig, für die eine Wiederverwendung möglich
erscheint oder sogar bereits geplant ist.

Begründen Sie für welche Komponenten eine Wiederverwendung sinnvoll erscheint
und für welche nicht!

Fügen Sie so viele Statechartdiagramme ein, wie sie Komponenten gefunden haben.
