
%Diese Datei dient der Versionskontrolle. Sie ist vollständig zu bearbeiten.

%----Überschrift------------------------------------------------------------
{\relsize{2}\textbf{Versionsübersicht}}\\[2ex]

%----Start der Tabelle------------------------------------------------------
\begin{longtable}{|m{1.78cm}|m{1.59cm}|m{2.86cm}|m{1.9cm}|m{5.25cm}|}

  \hline                                              % Linie oberhalb

  %----Spaltenüberschriften------------------------------------------------
  \textbf{Version}  &    \textbf{Datum}  &    \textbf{Autor}  &
  \textbf{Status}   &    \textbf{Kommentar}       \\  %Spaltenüberschrift
  \hline                                              % Gitterlinie

  %----die nachfolgeden beiden Zeilen so oft wiederholen und die ... mit den
  %    entsprechenden Daten zu füllen wie erforderlich
  0.1&02.05.2011&Christian Mangelsdorf&in Bearbeitung&Initialisierung\\
  \hline
  0.2&04.05.2011&Simon Hahne, Robin Hoffman, Christian Mangelsdorf&in Bearbeitung&Komponentenspezifiktion\\
  \hline
  0.3&06.05.2011&Matthias Überheide&in Bearbeitung&Vorbereitung Funktionsanalyse\\
  \hline
  0.4&08.05.2011&Matthias Überheide, Daniel Bahn, Konstantin Birker, Robin Hoffman, Simon Hahne, Christian Mangelsdorf&in Bearbeitung&Sequenzdiagramme und Statcharts\\
  \hline
  0.5&09.05.2011&Matthias Überheide, Daniel Bahn, Robin Hoffman, Simon Hahne, Christian Mangelsdorf&in Bearbeitung&Änderungen gemäß Besprechung\\
  \hline
  0.6&10.05.2011&Matthias Überheide, Christian Mangelsdorf&in Bearbeitung&Schnittstellenspezifikation\\
  \hline
  0.7&11.05.2011&Matthias Überheide, Daniel Bahn, Konstantin Birker, Robin Hoffman, Simon Hahne, Christian Mangelsdorf, Marco Melzzer&abgenommen&Korrekturen\\
  \hline
  %...    &    ...    &    ...    &    ...    &    ...\\       % Eintrag in Zeile
  \hline                                              % Gitterlinie unten

%----Ende der Tabelle------------------------------------------------------
\end{longtable}

%Status: "`in Bearbeitung"' oder "`abgenommen"'
%Kommentar: hier eintragen, was geändert bzw. ergänzt wurde


%Hinweis zum Template:
%Dieses Template enthält Hinweise, die alle kursiv geschrieben sind. Alles
%Kursivgeschriebene ist selbstverständlich bei Abgabe zu entfernen sind.
%Angaben in <\ldots> sind mit dem entsprechendem Text zu füllen.  Überzählige
%Kapitel, d. h. Kapitel, die nicht bearbeitet werden müssen, da sie nicht der
%Aufgabenstellung entsprechen, bitte entfernen.

%Aufgabe des Grobentwurfs: Aufgabe dieses Dokumentes ist es, die Architektur des
%Systems zu beschreiben und die daraus resultierenden Pakete durch die
%Definition von Schnittstellen zu Komponenten auszubauen.
