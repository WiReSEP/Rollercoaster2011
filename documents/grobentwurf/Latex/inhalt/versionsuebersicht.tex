%Diese Datei dient der Versionskontrolle. Sie ist vollständig zu bearbeiten.

%----Überschrift------------------------------------------------------------
{\relsize{2}\textbf{Versionsübersicht}}\\[2ex]

%----Start der Tabelle------------------------------------------------------
\begin{longtable}{|m{1.78cm}|m{1.59cm}|m{2.86cm}|m{1.9cm}|m{5.25cm}|}

  \hline                                              % Linie oberhalb

  %----Spaltenüberschriften------------------------------------------------
  \textbf{Version}  &    \textbf{Datum}  &    \textbf{Autor}  &
  \textbf{Status}   &    \textbf{Kommentar}       \\  %Spaltenüberschrift
  \hline                                              % Gitterlinie

  %----die nachfolgeden beiden Zeilen so oft wiederholen und die ... mit den
  %    entsprechenden Daten zu füllen wie erforderlich
  ...    &    ...    &    ...    &    ...    &    ...\\       % Eintrag in Zeile
  \hline                                              % Gitterlinie unten

%----Ende der Tabelle------------------------------------------------------
\end{longtable}

Status: "`in Bearbeitung"' oder "`abgenommen"'
Kommentar: hier eintragen, was geändert bzw. ergänzt wurde


Hinweis zum Template:
Dieses Template enthält Hinweise, die alle kursiv geschrieben sind. Alles
Kursivgeschriebene ist selbstverständlich bei Abgabe zu entfernen sind.
Angaben in <\ldots> sind mit dem entsprechendem Text zu füllen.  Überzählige
Kapitel, d. h. Kapitel, die nicht bearbeitet werden müssen, da sie nicht der
Aufgabenstellung entsprechen, bitte entfernen.

Aufgabe des Grobentwurfs: Aufgabe dieses Dokumentes ist es, die Architektur des
Systems zu beschreiben und die daraus resultierenden Pakete durch die
Definition von Schnittstellen zu Komponenten auszubauen.
