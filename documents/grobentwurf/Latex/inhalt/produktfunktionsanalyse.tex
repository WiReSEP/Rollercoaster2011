% Kapitel 2 mit den entsprechenden Unterkapiteln
% Die Unterkapitel können auch in separaten Dateien stehen,
% die dann mit dem \include-Befehl eingebunden werden.
%-------------------------------------------------------------------------------
\chapter{Analyse der Produktfunktionen}

Dieser Abschnitt stellt die Basis für die Festlegung der Architektur dar. Die
Festlegung einer geeigneten Architektur geschieht aufgrund der im Pflichtenheft
analysierten Produktfunktionen und nicht-funktionalen Anforderungen, die
realisiert werden müssen. Jede betrachtete Funktion wird in einem eigenen
Unterkapitel dokumentiert.  Fügen Sie bitte so viele Unterkapitel ein, wie
Produktfunktionen im Pflichtenheft vorhanden sind. Auch die nicht-funktionalen
Anforderungen sind so weit möglich entsprechend darzustellen.


\section{Analyse von Funktionalität <ID aus Pflichtenheft>: <Funktionsname>}
z. B.: Analyse von Funktionalität /F10/: Automatisches Einlagern
In diesem Abschnitt wird die im Titel angegebene Produktfunktion sowohl im
Hinblick auf ihre Verteilung auf die Architektur als auch im Hinblick auf die
zu ihrer Realisierung nötigen Datenstruktur untersucht.  Zu Beginn die
Funktionalität kurz beschreiben.  Anschließend erfolgt die Darstellung der
Realisierung der Funktion als Interaktion von Komponenten des zu entwickelnden
Systems in einem Sequenzdiagramm. Das Diagramm bitte ebenfalls kurz
beschreiben.

\section{Analyse der Funktionalität /F100/: Spezifikation einlesen}
Das Einlesen einer Spezifikation aus einem vom Nutzer angegebenen Dateipfad
gliedert sich in mehrere Teilfunktionalutäten. Zunächst muss der Anwender den
Dateipfad auswählen. Dafür sollte ein Datei-Dialog zum Einsatz kommen.
Danach muss die ausgewählte Datei geöffnet, geprüft und geparst werden. Für
diese Aufgaben bietet sich eine spezialisierte Komponente an, um alle
Eigenheiten des Dateiformates von der Oberfläche und dem Simulator abzukapseln.
Der ausgelesene Datensatz nun zum Starten einer neuen Simulation verwendet,
dafür muss eine eventuell laufende Simulation beendet werden. Es bietet sich
an, diese Funktionen als Schnittstelle zu kapseln und die konkrete Implementierung
der Simulator-Komponente zu überlassen.

(Bild einfügen)
Sequenzdiagramm für den Geschäftsprozess "Spezifikation einlesen"
Dargestellt ist die Interaktion des Nutzers mit der grafischen Benutzeroberfläche
beim Öffnen einer Datei, das Auslesen der Datei und der Wechsel der Simulation
