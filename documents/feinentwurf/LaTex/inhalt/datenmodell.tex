% Kapitel 4 mit den entsprechenden Unterkapiteln
% Die Unterkapitel können auch in separaten Dateien stehen,
% die dann mit dem \include-Befehl eingebunden werden.
%-------------------------------------------------------------------------------
\chapter{Datenmodell}
Falls in der Anwendung bestimmte Daten dauerhaft gespeichert werden, so sind
die entsprechenden Entities und Beziehungen hier darzustellen und zu erläutern.
Dies ist insbesondere relevant, falls der Einsatz einer (relationalen)
Datenbank geplant ist.

\section{Diagramm}

Eigenes ER-Modell einsetzen
\section{Erläuterung}
Die Tabelle ist um so viele Einträge zu erweitern, wie es Entities im obigen
ER-Modell gibt. Für jedes Entity sind so viele Einträge in der
Beziehungs-Subtabelle einzufügen, wie es Beziehungen zu diesem Entity gibt.


\begin{tabular}[ht]{|l|c|}
  \hline
  Entität & Beziehungen\\
  \hline\hline
  <Entity … ID>:  & Name der Beziehung |  Kardinalität\\
  \hline\hline\hline
  <Bezeichnung> & <Name der Beziehung> | <Kardinalität>\\
  \hline
\end{tabular}
