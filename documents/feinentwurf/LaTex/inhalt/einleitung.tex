% Kapitel 1
%-------------------------------------------------------------------------------

\chapter{Einleitung}
%Hier Einleitungstext einfügen, dabei die Formatierungen selber erstellen

Dieses Dokument erläutert den Entwurf für die Implementierung des Achterbahn-Simulators.
Gegenüber früheren Dokumenten liegt nun der Schwerpunkt nicht mehr auf den Konzepten 
der Domäne ''Achterbahn'' sondern bei der Umsetzung in der Zielsprache Java. 

Der höhere Detailierungsgrad des Feinentwurfs deckt Probleme bei der Umsetzbarkeit
der Anforderungen des Pflichtenheftes auf. Eine sorgfältige Prüfung findet sich im 
Kapitel ''Kriterienerfüllung''.

Entsprechend detailiert werden die Klassen mit ihren Attributen und Operationen auch innerhalb der
Komponenten spezifiziert. Um die Wartbarkeit, Erweiterbarkeit und Wiederverwertbarkeit
der Programmteile zu verbessern, wird auf den Einsatz von bewährten Entwurfsmustern
Wert gelegt; entsprechende Hinweise finden sich bei den Erläuterungen im Kapitel 
''Implementierungsentwurf''.

Auch wenn der Achterbahnsimulator ohne eine Datenbank arbeitet, wird das XML-Schema
für den Austausch der Achterbahndaten als persistente Komponente ausführlich 
im Kapitel ''Datenmodell'' erläutert.

