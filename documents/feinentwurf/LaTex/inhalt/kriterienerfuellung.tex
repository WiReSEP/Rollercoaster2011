% Kapitel 2 mit den entsprechenden Unterkapiteln
% Die Unterkapitel können auch in separaten Dateien stehen,
% die dann mit dem \include-Befehl eingebunden werden.
%-------------------------------------------------------------------------------
\chapter{Erfüllung der Kriterien}

Nachfolgend wird beschrieben, wie die einzelnen Kriterien des Pflichtenheftes
erfüllt werden und worauf geachtet wird.  Es ist dabei explizit auf die
definierten Kriterien des Pflichtenheftes zu verwiesen.
\section{Musskriterien}

Die folgenden Kriterien sind unabdingbar und müssen durch das Produkt erfüllt
werden:
\begin{description}
	\item[/M10/] Die physikalisch korrekte Wiedergabe des Fahrverhaltens der Achterbahn aus der Perspektive eines Fahrgastes wird durch eine in Echtzeit modellierte Bahn und eine Umgebung visualisiert. Es wird darauf geachtet, dass der Boden bei z=0 ist.
	\item[/M20/] Berechnung und Darstellung der Bahn erfolgt in Echtzeit. Es werden Optionen(teilweise Ausblendung von Dekoration) implementiert, die das System skalierbar machen und Beschleunigungsstrukturen genutzt.
	\item[/M30/] Zur Orientierung im 3DRaum wird ein texturierter Boden als Standartdekoration geladen. Durch Laden einer anderen Dekoration wird dieser Boden ersetzt.
	\item[/M40/] Aus den Kurvendaten der bezierkurve wird prozedural eine Bahn beliebigen Querschnitss erzeugt. Querstreben und stützen werden nach festen Regeln eingefügt.
	\item[/M50/] Eine 2D- Visualisierung der Beschleunigung wird in der 2D-GUI angezeigt und wird in einem Textfeld ausgewertet.
\end{description}

\section{Wunschkriterien}
Die Erfüllung folgender Kriterien für das abzugebende Produkt wird angestrebt:

\begin{description}
	\item[/W10/] Berechnung der Luft- und Bahnreibung.
	\item[/W20/] Erfassung von Bremsvorgängen.
	\item[/W30/] Berücksichtigung der Beschleunigungskräfte durch Wagen und Passagiere.
	\item[/W40/] Bei der Visualisierung von Stützbalken ist darauf zu Achten, dass keine Stützbalken durch die Bahn gehen. Dafür wird ein zusätzliches Boundingvolumen im Achterbahnobjekt vorgesehen, dass eine Kollisionsprüfung gegen die dynamisch erzeugt Bahn erlaubt.
	\item[/W50/] Die Darstellung der Umgebung mit Gebäuden, Bepflanzung und Himmel verschönern die Simulation und bieten dem Benutzer außerdem Orientierung. Wichtig dabei ist, dass keine Dekoration durch die Bahn geht.
	\item[/W60/] Die Verstellbare Kamera zwischen Innenansicht und Außenperspektive wird über eine Scrollbar in der 2D-GUI eingestellt und sofort geändert.
	\item[/W70/] Der Achterbahnwagen wird modelliert und ist bei der Außenperspektive sichtbar. Bei der Innenansicht ist der Wagen nicht zu sehen.
	\item[/W80/] Implementierung einer Zeitrafferfunktion mit der Möglichkeit zum Vor- und Zurückspulen wird über die 2D-GUI eingestellt.
	\item[/W90/] Implementierung einer Aufnahmefunktion für Einzelbilder und Filme wird im Hintergrund gestartet und speichert den fertigen Film automatisch ab.
\end{description}

\section{Abgrenzungskriterien}
Folgende Funktionalitäten werden nicht durch das Produkt, sondern wie folgt
beschrieben anderweitig erfüllt:
\begin{description}
	\item[/A10/] Keine eigenständige Funktionalität zum Erstellen oder Ändern der Achterbahnen. Diese Funktionalität wird von einem externen Editor geleistet.
	\item[/A20/] Keine Simulation der statischen und dynamischen Belastungen auf das Achterbahntragwerk
	\item[/A30/] Keine Überprüfung der Tragfähigkeit des Gerüstes
	\item[/A40/] Keine Berücksichtigung äußerlicher Einflüsse auf die Achterbahn
	\item[/A50/] Keine Passagiere
	\item[/A60/] Keine Untersuchung von Sicherungsmaßnahmen
\end{description}