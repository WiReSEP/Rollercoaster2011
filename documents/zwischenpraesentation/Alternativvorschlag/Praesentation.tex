%% Based on a TeXnicCenter-Template by Gyorgy SZEIDL.
%%%%%%%%%%%%%%%%%%%%%%%%%%%%%%%%%%%%%%%%%%%%%%%%%%%%%%%%%%%%%

%----------------------------------------------------------
%

\documentclass[draft]{beamer}%
\makeatletter
\def\Hy@xspace@end{}
\makeatother
%\let\ifGm@compatii\relax\makeatother
%
%----------------------------------------------------------%

\usetheme{Warsaw}
\setbeamercovered{transparent}

\usepackage{ngerman}						%%Deutsche Sprachdatei
\usepackage[latin1]{inputenc}		%%Umlaute
\usepackage[T1]{fontenc}				%%neue Kodierung

\usepackage{amsmath}%
\usepackage{amsfonts}%
\usepackage{amssymb}%
\usepackage{graphicx}

\title[SEP 2011]{Simualtion von Achterbahnen}
\subtitle{SEP 2011}
\author{Autor}
\institute[TU-BS]{Technische Universit�t Carolo-Wilhelmina zu Braunschweig}
\date{25.05.2011}


\begin{document}

	\begin{frame}
		\titlepage
	\end{frame}
	
	\begin{frame}{Gliederung}
		\begin{itemize}
			\item Physische Parameter / Mathematische Berechnungen
			\item 2D-Benutzeroberfl�che (GUI)
			\item 3D-Simulation
			\item Vorf�hrung des Prototypen
		\end{itemize}
	\end{frame}
	
	\section{Physik/Mathematik}
	\begin{frame}{Phyisk/Mathematik}
		\begin{itemize}
			\item Kurve wird �ber $e^{-i\pi}-1=0$ berechnet
			\item und ist ansonsten auch ganz toll
		\end{itemize}
	\end{frame}
	
	\section{GUI}
	\begin{frame}{GUI}
		\begin{itemize}
			\item immerhin ein Knopf
			\item mit toller Funktionalit�t
		\end{itemize}
	\end{frame}
	
	\section{3D-Anzeige}
	\begin{frame}{3D-Anzeige}
		\begin{itemize}
			\item eine selstame Achterbahn
			\item mit sch�ner Landschaft
			\item Automatische Kamerafahrt
		\end{itemize}
	\end{frame}
	
	\section{}
	\begin{frame}{Vorf�hrung}
		Prototy wird gestartet...\\
		bitte warten...
	\end{frame}

	
\end{document}
