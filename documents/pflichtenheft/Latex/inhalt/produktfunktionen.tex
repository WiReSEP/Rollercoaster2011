% Kapitel 4
%-------------------------------------------------------------------------------

\chapter{Produktfunktionen}

Im Nachfolgenden werden die Produktfunktionen näher beschrieben. Um optionale Funktionen zu Kennzeichnen sind die entsprechenden Prozesse mit einem zusätzlichen o im Kürzel bezeichnet.
\\ \\ 

/F100/\\
\textbf{Geschäftsprozess:} Spezifikation einlesen \\
\textbf{Ziel:} Ein Dateipfad wird angegeben und die neue Bahn der zu simulierenden Achterbahn wird der XML-Datei in allen Parametern entnommen. \\
\textbf{Vorbedingung:} Ein valider Dateiname wurde übergeben; die Simulation ist gestoppt oder pausiert. \\
\textbf{Nachbedingung Erfolg:} Die neue Bahn steht in allen Parametern dem Simulator im Speicher zur Verfügung. Der Ausgangszustand für eine Simulation wird hergestellt. \\
\textbf{Nachbedingung Fehlschlag:} Der Zustand vor dem Prozess ist wiederhergestellt. \\
\textbf{Akteure:} User \\
\textbf{Auslösendes Ereignis:} Userinteraktion \\
\textbf{Beschreibung:} \\
1 Der User wählt über einen Dateidialog eine XML Datei aus. \\
2 Das Programm interpretiert diese und liest die enthaltenen Paramter ein. Ein Syntaxcheck wird durchgeführt und ggf. eine Fehlermeldung angezeigt.\\
3 Die Daten werden in die Objekte des Simulators eingegeben. Fehlende Einträge werden soweit möglich mit Defaultwerten belegt.\\
\textbf{Erweiterungen:}\\
\textbf{Alternativen:}\\
\\
/F200/\\
\textbf{Geschäftsprozess:} Starten/Stoppen der Simulation\\
\textbf{Ziel:} Die Simulation der Achterbahn wird gestartet oder gestoppt.\\
\textbf{Vorbedingung:} Eine Achterbahnstrecke ist geladen.\\
\textbf{Nachbedingung Erfolg:} Der Status wird gewechselt (von gestartet auf gestoppt bzw. umgekehrt).\\
\textbf{Nachbedingung Fehlschlag:} Fehlermeldung; Simulation gestoppt.\\
\textbf{Akteure:} User\\
\textbf{Auslösendes Ereignis:} Userinteraktion\\
\textbf{Beschreibung:} Der User interagiert über eine Schaltfläche, die je nach aktuellem Status für die jeweilige Umschaltfunktion steht, und verändert damit den Simulationsstatus.\\
Das Stoppen führt zum Abbau von temporären Ergebnissen. Ein Fortsetzen der Simulation ist nicht mehr möglich (aber ein Neustart am Streckenanfang).\\
\textbf{Erweiterungen:}\\
\textbf{Alternativen:}\\
\\
/F300/\\
\textbf{Geschäftsprozess:} Pausieren der Simulation\\
\textbf{Ziel:} Die Simulation wird pausiert, kann jedoch zu einem späteren Zeitpunkt wieder fortgesetzt werden.\\
\textbf{Vorbedingung:} Eine Simulation läuft.\\
\textbf{Nachbedingung Erfolg:} Die Simulation ist unterbrochen.\\
\textbf{Nachbedingung Fehlschlag:} \\
\textbf{Akteure:} User\\
\textbf{Auslösendes Ereignis:} Userinteraktion\\
\textbf{Beschreibung:} \\
\textbf{Erweiterungen:}\\
\textbf{Alternativen:}\\
\\
/F400o/ \\
\textbf{Geschäftsprozess:} Video aufzeichnen\\
\textbf{Ziel:} Es wird ein Video der Visualisierung mit den aktuell zur Anzeige ausgewählten Informationen erzeugt und auf der Festplatte abgespeichert.\\
\textbf{Vorbedingung:} Eine Achterbahn ist geladen und die Simulation läuft nicht, die gewählte Zieldatei ist noch nicht existent, der Zielort ist beschreibbar.\\
\textbf{Nachbedingung Erfolg:} Es existiert eine Videodatei auf der Festplatte, die mit einem herkömmlichen Player abspielbar ist.\\
\textbf{Nachbedingung Fehlschlag:} Es existiert eine Datei auf der Festplatte, die die Daten der Aufzeichnung soweit wie möglich erhält.\\
\textbf{Akteure:} User\\
\textbf{Auslösendes Ereignis:} Userinteraktion\\
\textbf{Beschreibung:}  Der User wählt einen Zielort und das Video der Fahrt bis wird bis zum Ende der Strecke aufgezeichnet.\\
\textbf{Erweiterungen:}\\ Falls die Simulation läuft, wird sie gestoppt.
\textbf{Alternativen:}\\
\\
/F500/\\
\textbf{Geschäftsprozess:} Einstellungen ändern\\
\textbf{Ziel:} Teile der Einstellungen der Simulationsumgebung werden verändert.\\
\textbf{Vorbedingung:} \\
\textbf{Nachbedingung Erfolg:} Die Änderungen in den Einstellungen wurden umgesetzt.\\
\textbf{Nachbedingung Fehlschlag:}  Die Änderungen wurden mit einer Fehlermeldung zurückgewiesen.\\
\textbf{Akteure:} User\\
\textbf{Auslösendes Ereignis:} Userinteraktion\\
\textbf{Beschreibung:} \\
\textbf{Erweiterungen:}\\
\textbf{Alternativen:}\\
\\
/F510o/\\
\textbf{Geschäftsprozess:} Physikalische Paramter anpassen\\
\textbf{Ziel:} Parameter der physikalischen Berechnung werden angepasst und verändern damit den Fortlauf der Simulation.\\
\textbf{Vorbedingung:} Eine Achterbahn ist geladen.\\
\textbf{Nachbedingung Erfolg:} Die Änderungen wurden gespeichert und eine etwaig laufende Simulation pausiert.\\
\textbf{Nachbedingung Fehlschlag:} Die Änderungen wurden mit einer Fehlermeldung zurückgewiesen.\\
\textbf{Akteure:} User\\
\textbf{Auslösendes Ereignis:} Userinteraktion\\
\textbf{Beschreibung:} \\
\textbf{Erweiterungen:} Der User kann in einem Dialog gefragt werden, ob er wegen der neuen physikalischen Grundlagen die Simulation komplett neustarten möchte.\\
\textbf{Alternativen:}\\
\\
/F511o/\\
\textbf{Geschäftsprozess:} Gravitation anpassen\\
\textbf{Ziel:} Die Berechnung wird mit einer anderen Gravitationskonstante fortgesetzt.\\
\textbf{Vorbedingung:} Eine Achterbahn ist geladen.\\
\textbf{Nachbedingung Erfolg:} Die Änderungen wurden gespeichert und eine etwaig laufende Simulation pausiert.\\
\textbf{Nachbedingung Fehlschlag:} Die Änderungen wurden mit einer Fehlermeldung zurückgewiesen.\\
\textbf{Akteure:} User\\
\textbf{Auslösendes Ereignis:} Userinteraktion\\
\textbf{Beschreibung:} \\
\textbf{Erweiterungen:} Der User kann in einem Dialog gefragt werden, ob er wegen der neuen physikalischen Grundlagen die Simulation komplett neustarten möchte.\\
\textbf{Alternativen:}\\
\\
/F512o/\\
\textbf{Geschäftsprozess:} Wagenmasse anpassen\\
\textbf{Ziel:} Die Masse des Wagens wird mit dem neuen Wert angenommen und verändert das Simulationsergebnis (nur bei Reibung relevant).\\
\textbf{Vorbedingung:} Eine Achterbahn ist geladen.\\
\textbf{Nachbedingung Erfolg:} Die Änderungen wurden gespeichert und eine etwaig laufende Simulation pausiert.\\
\textbf{Nachbedingung Fehlschlag:} Die Änderungen wurden mit einer Fehlermeldung zurückgewiesen.\\
\textbf{Akteure:} User\\
\textbf{Auslösendes Ereignis:} Userinteraktion\\
\textbf{Beschreibung:} \\
\textbf{Erweiterungen:} Der User kann in einem Dialog gefragt werden, ob er wegen der neuen physikalischen Grundlagen die Simulation komplett neustarten möchte.\\
\textbf{Alternativen:}\\
\\
/F520/\\
\textbf{Geschäftsprozess:} Simulationsparameter ändern\\
\textbf{Ziel:} Paramete,r die den Ablauf der Simulation bestimmen, werden verändert und die Simulation wird fortgesetzt.\\
\textbf{Vorbedingung:}  \\
\textbf{Nachbedingung Erfolg:} Neue Einstellungen sind übernommen.\\
\textbf{Nachbedingung Fehlschlag:} Alte Einstellungen bleiben intakt.\\
\textbf{Akteure:} User\\
\textbf{Auslösendes Ereignis:} Userinteraktion\\
\textbf{Beschreibung:} \\
\textbf{Erweiterungen:}\\
\textbf{Alternativen:}\\
\\
/F521o/\\
\textbf{Geschäftsprozess:} Dekorative Umgebung anpassen\\
\textbf{Ziel:} Die dekorative Umgebung wird erweitert bzw. reduziert.\\
\textbf{Vorbedingung:} \\
\textbf{Nachbedingung Erfolg:} Die neuen dekorativen Elemente werden in der Visualisierung angezeigt.\\
\textbf{Nachbedingung Fehlschlag:} Alte Einstellungen bleiben intakt.\\
\textbf{Akteure:} User\\
\textbf{Auslösendes Ereignis:} Userinteraktion\\
\textbf{Beschreibung:} Aus einigen Elementen kann gewählt werden, welche zur Anzeige kommen sollen. Bspw. können Geländetopologien oder zusätzliche Stützen aktivierbar sein.\\
\textbf{Erweiterungen:} Eine Modelldatei wird geladen, die dann als Dekorationselement eingefügt wird.\\
\textbf{Alternativen:}\\
\\
/F522o/\\
\textbf{Geschäftsprozess:} Simulatioszeit anpassen\\
\textbf{Ziel:} Die Geschwindigkeit der Simulation wird verändert.\\
\textbf{Vorbedingung:} \\
\textbf{Nachbedingung Erfolg:} Neue Geschwindigkeitseinstellung werden übernommen.\\
\textbf{Nachbedingung Fehlschlag:} Alte Geschwindigkeitseinstellung bleiben intakt.\\
\textbf{Akteure:} User\\
\textbf{Auslösendes Ereignis:} Userinteraktion\\
\textbf{Beschreibung:} Die Zeitschritte, mit denen die Physik berechnet wird, wird manipuliert.\\
\textbf{Erweiterungen:}\\
\textbf{Alternativen:}\\
\\
/F530/\\
\textbf{Geschäftsprozess:} Graphische Einstellungen ändern\\
\textbf{Ziel:} Das graphische Erscheinungsbild des Programms wird verändert.\\
\textbf{Vorbedingung:} \\
\textbf{Nachbedingung Erfolg:} Das Erscheinungsbild wurden nach den Wünschen des Users angepasst.\\
\textbf{Nachbedingung Fehlschlag:} Das alte Erscheinungsbild ist in Kraft.\\
\textbf{Akteure:} User\\
\textbf{Auslösendes Ereignis:} Userinteraktion\\
\textbf{Beschreibung:} \\
\textbf{Erweiterungen:}\\
\textbf{Alternativen:}\\
\\
/F531/\\
\textbf{Geschäftsprozess:}  Neuanordnung (Interface)\\
\textbf{Ziel:} Das graphische Erscheinungsbild der Programmoberfläche wird verändert.\\
\textbf{Vorbedingung:} \\
\textbf{Nachbedingung Erfolg:} Das Erscheinungsbild wurden nach den Wünschen des Users angepasst.\\
\textbf{Nachbedingung Fehlschlag:} Das alte Erscheinungsbild ist in Kraft.\\
\textbf{Akteure:} User\\
\textbf{Auslösendes Ereignis:} Userinteraktion\\
\textbf{Beschreibung:} Das Interface des Programms wird nach den Wünschen des Users verschoben.\\
\textbf{Erweiterungen:} Aus/Einblenden eines HUD\\
\textbf{Alternativen:}\\
\\
/F532/\\
\textbf{Geschäftsprozess:} Ein-/Ausblenden von Beschleunigungsdaten\\
\textbf{Ziel:} Das Fenster mit den Graphen der physikalischen Werte wird ein- bzw. ausgeblendet.\\
\textbf{Vorbedingung:} \\
\textbf{Nachbedingung Erfolg:} Das Graphenfenster ist ein- bzw. ausgeblendet.\\
\textbf{Nachbedingung Fehlschlag:}\\ 
\textbf{Akteure:} User\\
\textbf{Auslösendes Ereignis:} Userinteraktion\\
\textbf{Beschreibung:} \\
\textbf{Erweiterungen:}\\
\textbf{Alternativen:}\\
\\
/F533o/\\
\textbf{Geschäftsprozess:} Kameraperspektive ändern\\
\textbf{Ziel:} Die Perspektive der Kamera ändert sich zwischen unterschiedlichen Modi.\\
\textbf{Vorbedingung:} \\
\textbf{Nachbedingung Erfolg:} Eine neue Kameraperspektive ist aktiv, aus der die Bilder gerendert werden.\\
\textbf{Nachbedingung Fehlschlag:} Die alte Perspektive ist in Kraft.\\
\textbf{Akteure:} User\\
\textbf{Auslösendes Ereignis:} Userinteraktion\\
\textbf{Beschreibung:} Durch entsprechende Interaktion kann zwischen vorgefertigten Perspektiven umgeschaltet werden, sodass die Fahrt auch aus anderen Blickwinkeln betrachtet werden kann.\\
\textbf{Erweiterungen:}\\
\textbf{Alternativen:}\\
\\
% die restlichen <1000 ggf reservieren für weitere Interfacefunktionen
/F1000/\\
\textbf{Geschäftsprozess:} Warnung vor hoher Beschleunigung\\
\textbf{Ziel:} Der User wird von der Überschreitung von gefährlichen Beschleunigungsgrenzen in Kenntnis gesetzt.\\
\textbf{Vorbedingung:} Eine Fahrt wird simuliert.\\
\textbf{Nachbedingung Erfolg:} Eine Nachricht wurde ausgegeben.\\
\textbf{Nachbedingung Fehlschlag:} \\
\textbf{Akteure:} Simulation\\
\textbf{Auslösendes Ereignis:} Überschreitung von Grenzwerten\\
\textbf{Beschreibung:} Bei der überschreitung von gefährlichen Grenzwerten bei den Beschleunigungen warnt das Programm mit entsprechenden Effekten bzw. einer Nachricht.\\
\textbf{Erweiterungen:}\\
\textbf{Alternativen:}\\

/F1100/\\
\textbf{Geschäftsprozess:} Erkennung von Veränderungen an der Ursprungsdatei\\
\textbf{Ziel:} Eine Änderung(zB. durch den Editor) an der aktuell gelesenen Bahndatei wird erkannt und die Bahn neu geladen\\
\textbf{Vorbedingung:} Eine Bahn ist geladen.\\
\textbf{Nachbedingung Erfolg:} Die Bahn wird gemäß der neuen Daten aufgebaut. Es wird ein Versuch unternommen die aktuelle Position auf der Bahn zu recovern. Ggf wird die Simulation neu gestartet.\\
\textbf{Nachbedingung Fehlschlag:} Der User wird informiert, dass sich die zugrunde liegende Datei verändert hat das einlesen der neuen Daten aber fehlgeschlagen ist\\
\textbf{Akteure:} Dateisysteminformation\\
\textbf{Auslösendes Ereignis:} Veränderung der geladenen Datei\\
\textbf{Beschreibung:} Auf die Veränderung der für die aktuelle Bahn verwendeten Datei reagiert der Simulator mit erneutem einlesen der Daten um damit einen schnelleren Workflow zwischen Editor und Simulator zu erlauben. 
Fehlerhafte Dateien vernichten dabei nicht den aktuell im Speicher befindlichen Datenbestand.\\
\textbf{Erweiterungen:}\\
\textbf{Alternativen:}\\