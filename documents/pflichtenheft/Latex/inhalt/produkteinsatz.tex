% Kapitel 2 mit den entsprechenden Unterkapiteln
% Die Unterkapitel können auch in separaten Dateien stehen,
% die dann mit dem \include-Befehl eingebunden werden.
%------------------------------------------------------------------------------------

\chapter {Produkteinsatz}
Die Simulation soll dem technischen Entwickler der Achterbahnen als Visualisierungswerkzeug für die physikalischen Gegebenheiten
dienen. Dadurch können durchgeführte Änderungen im Editor sofort optisch dargestellt und interpretiert werden.
Desweiteren soll die Simulation auch als Präsentationswerkzeug dienen.


\section {Anwendungsbereiche}
Kontruktionsbereich, technischer Entwicklungsbereich

\section{Zielgruppen}
Ingenieure, Präsentationspublikum

\section{Betriebsbedingungen}
Es handelt sich um eine Einzelplatzanwendung, die an einem Bürorechner oder Notebook betrieben werden kann.
Die Anwendung sollte mit einer 2 Jahre alten Grafikkarte mit 3D Beschleunigung lauffähig sein.
Der Simulator soll mit dem existierenden Editor zusammenspielen und auch gleichzeitig betrieben werden können.
