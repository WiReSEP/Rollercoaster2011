
%Diese Datei dient der Versionskontrolle. Sie ist vollständig zu bearbeiten. 

%----Überschrift------------------------------------------------------------
{\relsize{2}\textbf{Versionsübersicht}}\\[2ex]

%----Start der Tabelle------------------------------------------------------
\begin{longtable}{|m{1.78cm}|m{1.59cm}|m{2.86cm}|m{1.9cm}|m{5.25cm}|}

  \hline                                              % Linie oberhalb

  %----Spaltenüberschriften------------------------------------------------
  \textbf{Version}  &    \textbf{Datum}  &    \textbf{Autor}  &
  \textbf{Status}   &    \textbf{Kommentar}  \\  %Spaltenüberschrift
  \hline                                              % Gitterlinie

  %----die nachfolgeden beiden Zeilen so oft wiederholen und die ... mit den
  %    entsprechenden Daten zu füllen wie erforderlich
0.1&16.04.11&Robin Hofmann&in Bearbeitung&Grundgerüst\\
\hline 
0.2&16.04.11&Simon Hahne&in Bearbeitung&nichtfunktionale Anforderungen\\
\hline 
0.3&16.04.11&Robin Hofmann&in Bearbeitung&GUI\\
\hline 
0.4&17.04.11&Christian Mangelsdorf&in Bearbeitung&Zielbestimmung\\
\hline 
0.5&17.04.11&Matthias Überheide&in Bearbeitung&Produktübersicht und -funktionen\\
\hline 
0.6&17.04.11&Daniel Bahn&in Bearbeitung&Produkteinsatz\\
\hline 
0.7&17.04.11&Konstantin Birker&in Bearbeitung&Produktdaten\\
\hline 
0.8&18.04.11&alle&in Bearbeitung&Ergänzungen und Rechtschreibung\\
\hline 
0.9&19.04.11&Konstantin Birker&in Bearbeitung&Technische Produktumgebung\\
\hline 
0.10&20.04.11&Marco Melzer&in Bearbeitung&Glossar\\
\hline 
1.0&20.04.11&alle&abgenommen&\\
\hline 

  %...    &    ...    &    ...    &    ...    &    ...\\       % Eintrag in Zeile
  %\hline                                              % Gitterlinie unten

%----Ende der Tabelle------------------------------------------------------
\end{longtable}
%Status: "`in Bearbeitung"' oder "`abgenommen"'
%Kommentar: hier eintragen, was geändert bzw. ergänzt wurde



%Hinweis zum Template:
%Dieses Template enthält Beispiele und andere Hinweise, die alle kursiv
%geschrieben sind. Alles Kursivgeschriebene ist selbstverständlich bei Abgabe zu
%entfernen sind.

%Angaben in <\ldots> sind mit dem entsprechendem Text zu füllen.

