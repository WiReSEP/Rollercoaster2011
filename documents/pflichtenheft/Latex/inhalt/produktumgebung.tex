% Kapitel 10
% Die Unterkapitel können auch in separaten Dateien stehen,
% die dann mit dem \include-Befehl eingebunden werden.
%-------------------------------------------------------------------------------

\chapter{Technische Produktumgebung}

\section{Software}
Betriebssystem: Linux.\\
Für die Erstellung und Bearbeitung der Achterbahnspezifikationen gibt es einen Achterbahneditor.\\


\section{Hardware}
Die Software soll auf einem Mittelklasse-PC mit 2 Jahre alter OpenGL-fähiger Grafikkarte laufen.


%\section{Orgware}


\newpage

\section{Produktschnittstellen}
Es existiert bereits ein Achterbahneditor, der die Achterbahnspezifikationen erstellt.\\
Der Datenaustausch läuft über eine schematisierte XML-Datei.

\begin{verbatim}
<?xml version="1.0" encoding="utf-8"?>
<xs:schema id="RollerCoaster" targetNamespace="xml-schema-rollercoaster.xsd" 
elementFormDefault="qualified" attributeFormDefault="unqualified" 
xmlns="xml-schema-rollercoaster.xsd" xmlns:xs="http://www.w3.org/2001/XMLSchema">
  <xs:element name="RollerCoaster">
    <xs:complexType>
      <xs:sequence>
        <xs:element ref="Track" minOccurs="1" maxOccurs="unbounded"/>
      </xs:sequence>
    </xs:complexType>
  </xs:element>
  <xs:element name="Track">
    <xs:complexType>
      <xs:sequence>
        <xs:element ref="General" minOccurs="1" maxOccurs="1"/>
        <xs:element ref="SimulationParameters" minOccurs="0" maxOccurs="1"/>
        <xs:element ref="PillarList" minOccurs="1" maxOccurs="1"/>
      </xs:sequence>
     </xs:complexType>
  </xs:element>
  <xs:element name="General">
    <xs:complexType>
      <xs:sequence>
           <xs:element name="Name" type="xs:string" minOccurs="0" maxOccurs="1"/>
           <xs:element name="Author" type="xs:string" minOccurs="0" maxOccurs="1"/>
           <xs:element name="CreationDate" type="xs:string" minOccurs="0" maxOccurs="1"/>
           <xs:element name="Length" type="xs:double" minOccurs="0" maxOccurs="1"/>
           <xs:element name="Comment" type="xs:string" minOccurs="0" maxOccurs="1"/>
      </xs:sequence>
    </xs:complexType>
  </xs:element>
\end{verbatim}

%Zeilenumbruch von Hand (nicht schoen aber selten)
\newpage

\begin{verbatim}
   <xs:element name="SimulationParameters">
    <xs:complexType>
      <xs:sequence>
           <xs:element name="Speed" type="xs:double" minOccurs="1" maxOccurs="1"/>
           <xs:element name="Scale" type="xs:double" minOccurs="1" maxOccurs="1"/>
           <xs:element name="Direction" type="xs:string" minOccurs="1" maxOccurs="1"/>
      </xs:sequence>
    </xs:complexType>
  </xs:element>
  <xs:element name="PillarList">
    <xs:complexType>
      <xs:sequence>
        <xs:element ref="Pillar" minOccurs="0" maxOccurs="unbounded"/>
      </xs:sequence>
    </xs:complexType>
  </xs:element>
  <xs:element name="Pillar">
    <xs:complexType>
      <xs:sequence>
        <xs:element name="PosX" type="xs:double" minOccurs="1" maxOccurs="1"/>
        <xs:element name="PosY" type="xs:double" minOccurs="1" maxOccurs="1"/>
        <xs:element name="PosZ" type="xs:double" minOccurs="1" maxOccurs="1"/>
        <xs:element name="YawX" type="xs:double" minOccurs="1" maxOccurs="1"/>
        <xs:element name="YawY" type="xs:double" minOccurs="1" maxOccurs="1"/>
        <xs:element name="YawZ" type="xs:double" minOccurs="1" maxOccurs="1"/>
        <xs:element name="YawAngle" type="xs:double" minOccurs="1" maxOccurs="1"/>
      </xs:sequence>
    </xs:complexType>
  </xs:element>
</xs:schema>
\end{verbatim}

