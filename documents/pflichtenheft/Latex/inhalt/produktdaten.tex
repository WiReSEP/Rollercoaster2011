% Kapitel 5
%------------------------------------------------------------------------------------------

\chapter{Produktdaten}
5  Produktdaten

/D10/ Daten der Achterbahnkonstruktion:\\
-  Die Konstruktionsdaten der Achterbahnen werden aus dem Dateisystem eingelesen.\\
-  Die Daten werden ausschließlich gelesen und nicht geschrieben.//
-  Die Übergabe der Daten erfolgt durch eine schematisierte XML-Datei (Aufbau siehe Produktumgebung->Produktschnittstellen).//
-  Bestandteile:
--  geordnete Liste von Stützstellen (max. 1000)
--  je Stützpunkt
---  Position im 3D-System
---  Orientierung durch Giervektor im 3D-System

/D20/ Auswertungsdaten der Simulation:\\
-  Die Daten können optional im Dateisystem gespeichert werden.\\
-  Bestandteile:
--  Zeitpunkt,\\
--  Position,\\
--  Geschwindigkeit,\\
--  Beschleunigung/Kräfte,\\

/D30/ Videodaten
-  Optinal kann ein Video der Achterbahnsimulation im AVI-Format im Dateisystem abgelegt werden.//

/D40/ Bilddaten
-  Optional kann ein Screenshot im PNG-Format im Dateisystem abgelegt werden.