% Kapitel 5
%------------------------------------------------------------------------------------------

\chapter{Produktdaten}

/D10/ Daten der Achterbahnkonstruktion:
\begin{itemize}
\item Die Konstruktionsdaten der Achterbahnen werden aus dem Dateisystem eingelesen.
\item Die Daten werden ausschließlich gelesen und nicht geschrieben.
\item Die Übergabe der Daten erfolgt durch eine schematisierte XML-Datei (Aufbau siehe Produktumgebung->Produktschnittstellen).
\item Bestandteile:
	\begin{itemize}
	\item geordnete Liste von Stützstellen (max. 1000)
	\item je Stützpunkt
		\begin{itemize}
		\item Position im 3D-System
		\item Orientierung durch Giervektor im 3D-System
		\end{itemize}
	\end{itemize}
\end{itemize}

/D20/ Auswertungsdaten der Simulation:
\begin{itemize}
\item Die Daten können optional im Dateisystem gespeichert werden.
\item Bestandteile:
\begin{itemize}
\item Zeitpunkt,
\item Position,
\item Geschwindigkeit,
\item Beschleunigung/Kräfte,
\end{itemize}
\end{itemize}

/D30/ Videodaten
\begin{itemize}
\item  Optional kann ein Video der Achterbahnsimulation im AVI-Format im Dateisystem abgelegt werden.
\end{itemize}

/D40/ Bilddaten
\begin{itemize}
\item  Optional kann ein Screenshot im PNG-Format im Dateisystem abgelegt werden.
\end{itemize}
