% Kapitel 5
%------------------------------------------------------------------------------------------

\chapter{Produktdaten}
5  Produktdaten
Die langfristig zu speichernden Daten sind aus Benutzersicht detaillierter zu
beschreiben. Dabei bietet sich eine formale Beschreibung (z. B. in Form eines
Data Dictionary) an, um eine größere Präzisierung zu erreichen.

Bitte die Darstellung gemäß Beispiel verwenden:\\
-  /D10/: fortlaufende Nummerierung der Daten\\
-  /LD10/: vordefinierte Daten aus dem Lastenheft (falls vorhanden).\\

Beispiel: Lagerdaten\\
/D10/ (/LD10/) Daten der Lagerplätze (max. 5.000):\\
-  Modulnummer,\\
-  Regalseite,\\
-  Regalspalte,\\
-  Regalzeile,\\
-  Fachhöhe,\\
-  Platzsperre (0 = nicht gesperrt, 1 = gesperrt für Einlagerung, 2 = gesperrt
   für Auslagerung, 3 = gesperrt für alle Zugriffe),\\
-  Reifenstatus (0 = frei,1 = reserviert für Einlagerung, 2= belegt, 3 =
   reserviert für Auslagerung),\\
-  Reifenseriennummer.\\

/D40/ (/LD40/) Daten der Module (max. 20):\\
-  Modulnummer,\\
-  Sperrkennzeichen (0 = nicht gesperrt, 1 = gesperrt für Einlagerung, 2 =
   gesperrt für Auslagerung, 3 = gesperrt für alle Zugriffe),\\
-  maximale Kapazität,\\
-  freie Kapazität,\\
-  belegte Plätze (ergibt sich aus Status und Zahl der zugeordneten
   Lagerplätze, wird aus Geschwindigkeitsgründen allerdings redundant
   mitgeführt).

