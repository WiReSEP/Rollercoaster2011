% Kapitel 6 
%-------------------------------------------------------------------------------

\chapter{Nichtfunktionale Anforderungen}


%----Start der Tabelle------------------------------------------------------
\begin{tabular}{|c|c|c|c|c|}
  \hline                                              % Linie oberhalb

  %----Spaltenüberschriften------------------------------------------------
  \textbf{Produktqualität}  & \textbf{sehr gut}  &    \textbf{gut}  &
  \textbf{normal}           & \textbf{nicht relevant}  \\  %Spaltenüberschrift
  \hline                                              % Gitterlinie

  %----die nachfolgeden beiden Zeilen so oft wiederholen und die ... mit den
  %    entsprechenden Daten zu füllen wie erforderlich
  \textbf{Funktionalität}  &&&&\\        % Eintrag in Zeile
  \hline
Angemessenheit&&x&&\\
\hline
Richtigkeit&x&&&\\
\hline
Interoperabilität&&&x&\\
\hline
Ordnungsmäßgkeit&&&x&\\
\hline
\textbf{Sicherheit}&&&&\\
\hline
Zuverlässigkeit&&x&&\\
\hline
Reife&&x&&\\
\hline
Fehlertoleranz&&&x&\\
\hline
Wiederherstellbarkeit&&&x&\\
\hline
\textbf{Benutzbarkeit}&&&&\\
\hline
Verständlichkeit&&&x&\\
\hline
Erlernbarkeit&&x&&\\
\hline
Bedienbarkeit&x&&&\\
\hline
Effizienz&x&&&\\
\hline
Zeitverhalten&x&&&\\
\hline
Verbrauchsverhalten&&&x&\\
\hline
\textbf{Änderbarkeit}&&&&\\
\hline
Analysierbarkeit&&&x&\\
\hline
Modifizierbarkeit&&x&&\\
\hline
Stabilität&&&x&\\
\hline
Prüfbarkeit&&x&&\\
\hline
Übertragbarkeit&&&&\\
\hline
Anpassbarkeit&&&&x\\
\hline
Installierbarkeit&&x&&\\
\hline
Konformität&&&x&\\
\hline
Austauschbarkeit&&&&x\\
\hline
%----Ende der Tabelle------------------------------------------------------
\end{tabular}

\begin{itemize}

\item  /Q10/ Das Produkt soll in Java geschrieben sein.
\item  /Q20/ Das Produkt muss unter Linux laufen.
\item  /Q30/ Das Produkt soll auf einem Computer mit Open GL fähiger Grafikkarte laufen.
\item  /Q40/ Die Simulation soll in Echtzeit berechnet werden.
\item  /Q50/ Durch die 3D Grafik kann man vorher eine Stabilität nicht garantieren.

\end{itemize}
